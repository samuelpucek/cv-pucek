%%%%%%%%%%%%%%%%%%%%%%%%%%%%%%%%%%%%%%%%%
% Medium Length Professional CV
% LaTeX Template
% Version 2.0 (8/5/13)
%
% This template has been downloaded from:
% http://www.LaTeXTemplates.com
%
% Original author:
% Rishi Shah 
%
% Important note:
% This template requires the resume.cls file to be in the same directory as the
% .tex file. The resume.cls file provides the resume style used for structuring the
% document.
%
%%%%%%%%%%%%%%%%%%%%%%%%%%%%%%%%%%%%%%%%%

%----------------------------------------------------------------------------------------
%	PACKAGES AND OTHER DOCUMENT CONFIGURATIONS
%----------------------------------------------------------------------------------------

\documentclass{resume} % Use the custom resume.cls style

\usepackage{hyperref}

\usepackage[left=0.75in,top=0.6in,right=0.75in,bottom=0.6in]{geometry} % Document margins
\newcommand{\tab}[1]{\hspace{.2667\textwidth}\rlap{#1}}
\newcommand{\itab}[1]{\hspace{0em}\rlap{#1}}
\name{Samuel Pucek} % Your name
\address{Simunkova 1608/19, Prague 8, Czech Republic} % Your address
\address{+420 770 189 639 \\ \href{mailto:samuel.pucek@gmail.com}{samuel.pucek@gmail.com} \\ GitHub: \href{https://github.com/samuelpucek}{samuelpucek} }

 % Your phone number and email

\begin{document}

%--------------------------------------------------------------------------------
%    Carrier Objective
%-----------------------------------------------------------------------------------------------
\begin{rSection}{About Me}
I’m friendly and open-minded person, with a passion for technology. With strong mathematical background (MSc in STEM) I focus on statistics and programming. In my spare time I love running and hiking.

My daily routine consists of developing ML models and pipelines, creating ad-hoc analysis, and supporting other colleagues. I'm fluent in Python, SQL and Git. In my tasks I am systematic and persistent.
\end{rSection}

%----------------------------------------------------------------------------------------
%	WORK EXPERIENCE SECTION
%----------------------------------------------------------------------------------------
\begin{rSection}{Work Experience}
 
    \begin{rSubsection}{Gen (formerly Avast and NortonLifeLock)}{\em Prague, Czech Republic}{Principal Data Scientist}{Feb 2023 - Present}
        \item Experimentation and A/B testing
        
        I improve experimentation, specifically implement new metrics tailored to the business, model customer LTV and support business with various pre-experiment and post-experiment analyses (Jupyter Notebooks). I prototype variance reduction methods, i.e., stratification and CUPED.

    \end{rSubsection}

    \begin{rSubsection}{Avast (now Gen)}{\em Prague, Czech Republic}{Data Scientist}{Jan 2022 - Jan 2023}
        \item Architecture and development of ML pipelines
        
        I worked on architecture and development of Machine Learning pipelines which supported retention strategy and personalisation in Avast (Dynamic Discounting). I was responsible for end-to-end pipelines architecture and development, the quality of code delivered by third parties, communication with other internal and external teams, and support of other colleagues, including onboarding.
        
        In the Dynamic Discounting project were used XGBoost models followed by Reinforcement Learning model. The pipeline was scheduled daily, predictions were passed to other internal systems.

    \end{rSubsection}

    \begin{rSubsection}{Avast (now Gen)}{\em Prague, Czech Republic}{Junior Data Scientist}{Sep 2019 - Dec 2021}
        \item Avast's Experimentation Platform
        
        In Avast, we developed internally the Experimentation Platform, self-service tool for A/B testing. My main contribution to Experimentation Platform was to design a statistical engine for evaluating experiments, which consisted of sequential evaluation, multiple comparison correction, delta method etc.
        The solution was open-sourced in fall 2020, check GitHub repo \href{https://github.com/avast/ep-stats}{\texttt{ep-stats}}. The corresponding Python package can be found in \href{https://pypi.org/project/ep-stats/}{PyPi}.
        
        In \texttt{ep-stats} we mainly focused on computational complexity. Using \texttt{numpy} we vectorized many math operations. We back-upped all features using unit testing, including statistical features stated above. The project met all object-oriented programming standards.
        Next, I provided various deep-dive analyses (Jupyter Notebooks) to answer business-related questions, maintained and updated a few legacy data pipelines.

        \textit{After a year since the launch of the Experimentation Platform we served over 100 experiments per month. The experimentation mindset grew enormously; all ideas in the sales and marketing were A/B tested before being full scaled. EP provided evaluation of the experiments using a unified set of business metrics. The goal was to further evangelise experimentation mindset into engineering and product development.}
        
    \end{rSubsection}
        
    \begin{rSubsection}{O2 Czech Republic}{\em Prague, Czech Republic}{Internship and QA Tester (part-time)}{Aug 2016 - Aug 2019}
        \item IT reporting in SAP Solution Manager
        \item Integration and regression testing of the global CRM system (SQL, HP Quality Center, SAP)
    \end{rSubsection}
        
\end{rSection}
    
% \newpage
%----------------------------------------------------------------------------------------
%	TECHNICAL STRENGTHS SECTION
%----------------------------------------------------------------------------------------
\begin{rSection}{Technical Strengths}
    \begin{tabular}{ @{} >{\bfseries}l @{\hspace{6ex}} l }
    Programming		& Python (Object-oriented programming, Jupyter Notebooks) \\
    Libraries		& Pandas, NumPy, Plotly, scikit-learn \\
    Software \& Tools 		& SQL, Airflow, Git, \LaTeX \\
    Data             		& GCP (cloud), Hadoop (on prem), Hive, ClickHouse \\
    \end{tabular}
    \end{rSection}

%----------------------------------------------------------------------------------------
%	EDUCATION SECTION
%----------------------------------------------------------------------------------------
\begin{rSection}{Education}
    {\bf Charles University in Prague, Faculty of Mathematics and Physics} \hfill {\em 2014 - 2019} 
    \\ Master's degree in Probability, Mathematical Statistics and Econometrics
    \end{rSection}
    
%----------------------------------------------------------------------------------------
%	THESES SECTION
%----------------------------------------------------------------------------------------
%\begin{rSection}{Theses}
%    Master thesis: {\bf Risk aversion in portfolio efficiency} \hfill {\em Sep 2019}\\
%    Bachelor thesis: {\bf Scheduling optimization problems in education} \hfill {\em Sep 2017}
%    \end{rSection}

%----------------------------------------------------------------------------------------
%	LANGUAGES
%----------------------------------------------------------------------------------------
\begin{rSection}{Languages}
\begin{tabular}{ @{} >{\bfseries}l @{\hspace{6ex}} l }
English	& Professional working proficiency\\
Slovak	& Native
\end{tabular}
\end{rSection}

%--------------------------------------------------------------------------------
%    Projects And Seminars
%-----------------------------------------------------------------------------------------------
\begin{rSection}{Projects}
{\bf Blackjack with RL agent} \hfill {\em Dec 2022}\\
I coded Blackjack game in Python with standard actions, i.e. hit, stand, double down, split, surrender and insurance. The goal is to add Reinforcement Learning agent on top of the game. The agent should learn how to play Blackjack on its own, and wisely use strategy when the game is "hot".\\
See the \href{https://github.com/samuelpucek/blackjack}{\texttt{blackjack}} repo in my personal GitHub.
% {\bf Mini kiwi.com} \hfill {\em Jun 2019}\\
% I took part on Python Weekend in Bratislava organized by kiwi.com. General topic was to create a~web application for searching ground connections (buses and trains). User fill web application parameters, choose source and destination city and date. The program will search all available connections. Offered journeys may contain multiple operators, e.g. FlixBus, České dráhy, RegioJet,~... We were dealing with both front and back end. We literally started on a greenfield. Schedules were gained from operators web pages via various requests and apis. Redis was used for optimization. All data were stored in SQL database for later re-usage. The whole project was coded in Python 3.6.\\
% {\bf Discrete simulation of music festival} \hfill {\em Feb 2019}\\
% The simulation was coded in C\#, with focus on user-friendly interface.\\
% User can choose number of visitors, service speed (e.g. beer, hot-dog, toilets), and couple of other parameters. Program itself randomly creates visitors with various preferences and their own music schedule. Then the simulation starts. Randomness is also included in decisions during the simulation. The output is the overall satisfaction of visitors and detailed history of all visitors.\\
% \\{\bf Modeling mortgage rates} \hfill {\em Oct 2018 - Jan 2019}\\
% This project aimed at developing a framework for predicting mortgage rates for the purpose of actuarial cash flow models implemented in insurance company. We suggested multiple approaches, i.e. vector autoregression, simple regression. Bootstrap algorithm was used for generating scenarios. In addition, we dealt with client specific data of the insurance company. We used R and EViews for regression. Bootstrap and simulations were made in MATLAB. Client specific data were analysed in R.\\
% \\{\bf Primary school timetable} \hfill {\em Aug 2018}\\
% I used the results obtained in my Bachelor thesis and implemented them in real-life problem. Using GAMS I solved large linear optimization problem, which covered all requirements and conditions. The output was optimal timetable for the upcoming academic year. R was used for dealing with data before, and after the optimization.
\end{rSection}

%----------------------------------------------------------------------------------------
%	HOBBY SECTION
%----------------------------------------------------------------------------------------
\begin{rSection}{Individual interests}
Running (marathon 3:00:03, half-marathon 1:23:51), hiking, cycling.
\end{rSection}

\end{document}
